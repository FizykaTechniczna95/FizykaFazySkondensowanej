\subsubsection*{Rozwiązanie:}
\textbf{przypadek 1}
Stała Madelunga jest używana do wyznaczania potencjału elektrostatycznego pojedynczych jonów w krysztale, przez zbliżanie się jonów przez ładunki punktowe. 
\\
\\
wzór ogólny:
\\
\\
\begin{equation}
\alpha=\sum \frac {\pm}{p_i}
\end{equation}
\\
\\
W naszym przypadku: 
\\
\\
\begin{equation}
\frac{\alpha}{R}=2(\frac{1}{R}-\frac{2}{R}+\frac{1}{3R})-\frac{1}{4R}+\frac{1}{5R}
\end{equation}
\\
gdzie jako jon odniesienia bierzemy jon zaznaczony na rysunku, następnie odejmujemu lub dodajemy w zależności od znaku jonu w minaowniku są wielokrotności odległości od jonu odniesienia.
\\
\\
używając rozwinięcia w szereg Maclurina możemy udowodnić, że szereg:
\\
\\
\begin{equation}
x-\frac{x^2}{2}+\frac{x^3}{3}-\frac{x^4}{4}....
\end{equation}
\\
\\
którego wzór ogólny ma postać:
\\
\\
\begin{equation}
\sum_{n=1}^{\infty}\frac{(-1)^{n+1}}{n}x^n
\end{equation}
\\
\\
jest zbierzny do funkcji ln(1+x), gdzie x=1.
\\
\\
Rozwinięcie:
\\
\\
$f(x)=ln(x+1)$   =>   $f(0)=ln(1)=0$
\\
\\
$f'(x)=-\frac{1}{(x+1)^2}$   =>   $f(0)=-1$
\\
\\
$f''(x)=\frac{2}{(x+1)^3}$   =>   $f(0)=2$
\\
\\
$f'''(x)=-\frac{6}{(x+1)^5}$   =>   $f(0)=-6$
\\
\\
$=0+1*\frac{x}{1}-1*\frac{x^2}{2}+\frac{x^3}{6}-\frac{x^4}{4}$
\\
\\
ogólny wynik:
$\alpha=2ln2$
\\
\\
\textbf{przypadek drugi}
\\
\\
Odległości obliczmy w natępujący sposób:
\\
\\
-jako jon odniesienia bierzemy jon w środku.
\\
\\
-dzielimy przestrzeń na cztery kwadraty i wszystko liczmy dla jednego kwadratu potem mnożymy razy cztery.
\\
\\
-pierwsze trzy wartości w nawiasie są to odległości od jonu odniesienia w prawo (liczmy je tak jak poprzednio, w liczniku znak, a wmianowniku wielokrotność odległości od jonu odniesienia)
\\
\\
-następnie liczmy odległość jonu oznaczone gwiazdką, mnożymy przez odpowiedni znak i dodajemy do sumy (liczmy ze wzoru Pitagorasa)
\\
\\
następnie liczmy odległości dwóch jonów oznaczonych paprykami i mnożymy razy dwa ponieważ na prawo od jonu z gwiazdką też mamy papryki.
\\
\\
-potem liczymy jony oznaczone słońcami 
\\
\\
-a na końcu jon oznaczony prostokątem i mnożymy razy dwa bo po prawej też jest prostokąt.
\\
\\
I wychodzi nam:
\\
\\
\begin{equation}
\frac{\alpha}{R} =\frac{4}{R}(1-\frac{1}{2}+\frac{1}{3}-\frac{1}{\sqrt{2}}=\frac{2}{\sqrt{5}}-\frac{2}{\sqrt{10}}-\frac{1}{2\sqrt{2}}-\frac{1}{3\sqrt{2}}+\frac{2}{\sqrt{13}})
\end{equation}
\\
\\
opisane wyżej odległości, które znajdują się w licznikach liczymy ze wzoru pitagorasa, np. dla jonu z gwiazdką:
\\
\\
\begin{equation}
R^2+R^2=R^2
\end{equation}
\begin{equation}
2R^2=R^2
\end{equation}
\begin{equation}
R=\sqrt{2}R
\end{equation}
