\subsubsection*{Rozwiązanie:}
\textbf{funkcja gęstości stanów}
D$(\omega)$ jest to liczba modów różnych dragań przypadająca na jednostkowy zakres częstotliwości.
\\
\\
przemieszczenia atomu w drganiach podłuznych i poprzecznych określa zależność:
\\
\\
$U_s ~sin(sKa)$
\\
\\
dobieramy tak wartości K żeby atomy na końcu i początku łańcucha były unieruchomione.
\\
wartości wektora falowego, które są dozwolone:
\\
\\
K= +/-$\frac{2\pi}{L},\frac{4\pi}{L},.....$
\\
Gęstość stanów opisuje wzór:
\\
\\
\begin{equation}
D(\omega)=\frac{dN}{d\omega}
\end{equation}
\\
gdzie N to całkowita liczba modów drgań o wartości wektora falowego mniejszego od k.
\\
\\
$K=\frac{N\pi}{L}$
\\
\\
$N=\frac{KL}{N}$
\\
\\
$D(\omega)=\frac{d}{d\omega}\frac{KL}{\pi}=\frac{L}{\pi}\frac{dK}{d\omega}=\frac{L}{\pi v}=\frac{L K}{\pi \omega}$
\\
\\
gdzie $\frac{dK}{d\omega}$ - prędkość grupowa, $k=\frac{\omega}{v}$
\\
\\
częstotliwość Debaya jest to teoretyczna najwyższa możliwa częstotliwość drgań atmów  wsieci krystalicznej.
\\
\\
$N=\frac{KL}{\pi}=\frac{\omega L}{v\pi}$
\\
\\
$\omega_D=\frac{Nv\pi}{L}$


\hrulefill
