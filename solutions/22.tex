\subsubsection*{Rozwiązanie:}
\begin{enumerate}
\item Energia potencjalna oddziaływania między dwoma atomami dana jest przez:
\begin{equation}
\label{eq:UR}
U(R) = N\left( \frac{A}{R^n} - \frac{\alpha q^2}{R} \right)
\end{equation}
gdzie $\alpha$ jest stałą Madelunga wynoszącą dla przypadku liniowego $\alpha = 2\ln 2$.\\
Dla stanu równowagi spełniony musi być warunek:
\begin{equation}
\frac{\partial U}{\partial R} = 0
\end{equation}
stąd:
\begin{equation}
\frac{\partial U}{\partial R} = N \left( -\frac{nA}{R_0^{n+1}} + \frac{\alpha q^2}{R_0^2} \right)= 0
\end{equation}
\begin{equation}
\label{eq:dudr}
\frac{nA}{R_0^{n+1}} = \frac{\alpha q^2}{R_0^2} \rightarrow \frac{A}{R_0^n} = \frac{\alpha q^2}{n R_0}
\end{equation}
Wstawiając wyrażenie (\ref{eq:dudr}) do (\ref{eq:UR}) oraz $\alpha=2\ln(2)$ otrzymuje się:
\begin{equation}
U(R_0) = N \left( \frac{\alpha q^2}{R_0^n} - \frac{\alpha q^2}{R_0} \right) = \frac{2Nq^2\ln(2)}{R_0} \left( \frac{1}{n} -1 \right)
\end{equation}
\hrulefill
\item Rozwijając w szereg wokół $R_0$ ($\frac{\partial U}{\partial R} =0 $):
\begin{equation}
U(R_0-\delta R_0 ) = R_0 + \frac{1}{2} \frac{\partial^2 U}{\partial R^2} R_0 (\delta R_0)^2
\end{equation}
\end{enumerate}
