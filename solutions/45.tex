\subsubsection*{Rozwiązanie:}
Energia fermiego- energia najwyżeszego obsadzonego układu N elektronów:
\\
\\
\begin{equation}
E_f=\frac{\hbar^2}{2m}(\frac{n_f\pi}{L})^2
\end{equation}
\\
\\
$n_f$-najwyższy obsadzony poziom energii
\\
\\
przyjmując, że N jest liczbą parzystą: $2n_f=N$ otzrymujemy:
\\
\\
\begin{equation}
E_f=\frac{\hbar^2}{2m}(\frac{N\pi}{2L})^2
\end{equation}
\\
\begin{equation}
N=\frac{2L}{\hbar \pi}\sqrt{2mE_f}
\end{equation}
\\
\\
$D{\omega}$-gęstość stanów-liczba orbitali przypadająca na jednostkę energii.
\\
\\
\begin{equation}
\frac{dN}{d\omega}=L\frac{\sqrt{2m}}{\pi \hbar}\frac{1}{\sqrt{E_f}}
\end{equation}