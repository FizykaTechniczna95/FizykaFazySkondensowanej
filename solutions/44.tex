\subsubsection*{Rozwiązanie:}
Energia Fermiego dana jest wzorem:
\begin{equation}
\label{eq:E_F}
E_F = \frac{\hbar^2}{2m}k_F^2
\end{equation}
Na każdy element objętości $v_1 = \left(\frac{2\pi}{L}\right)^3$ przypada jeden wektor falowy $k$.\\
Stąd dla objętości $v_2 = \frac{4}{3}\pi k_F^3$ całkowita liczba stanów wynosi($2$ dozwolone stany spinowej liczby kwantowej):
\begin{equation}
N = 2 \frac{v_2}{v_1} = 2 \frac{4\pi k_F^3 L^3}{3\cdot 8 \pi^3} =  \frac{V}{3\pi^2}k_F^3
\end{equation}
czyli
\begin{equation}
\label{eq:k_F}
k_F = \left( \frac{3 \pi^2 N}{V} \right)^{\frac{1}{3}} =  (3\pi n)^{\frac{1}{3}}
\end{equation} 
gdzie gęstość stanów $n = \frac{N}{V}$.\\
Podstawiając wyrażenie (\ref{eq:k_F}) do równania (\ref{eq:E_F}) otrzymuje się:
\begin{equation}
E_F = \frac{\hbar^2}{2m} (2n\pi)^{\frac{2}{3}}
\end{equation} 
\hrulefill
\newline
W przypadku 1D energia elektronu w stanie $n$ wynosi:
\begin{equation}
E_n = \frac{\hbar^2 \pi^2}{2mL^2}n^2
\end{equation}
Dla  $N$ poziomów śrenią energię można wyrazić przez:
\begin{equation}
\overline{E} = 2 \frac{\sum^N E_n}{2N}
\end{equation}
gdzie energia ostatniego elektronu jest enegią Fermiego:
\begin{equation}
\label{eq:E_F2}
E_F = \frac{\hbar^2 \pi^2}{2mL^2} N^2
\end{equation}
Stosując przybliżenie:
\begin{equation}
\sum^N_{n=1} n^2 = \frac{1}{6}(2N^2 + 3N + 1) = \frac{N^3}{3} + \frac{N^2}{2} + \frac{N}{6} \sim \frac{N^3}{3}
\end{equation} 
oraz (\ref{eq:E_F2}), średnia energia elektronu wyraża się przez:
\begin{equation}
\overline{E} = \frac{2}{2N} \frac{\hbar^2 \pi^2}{2mL^2} \sum^N n^2 = \frac{2}{2N} \frac{\hbar^2 \pi^2}{2mL^2} \frac{N^3}{3} = \frac{1}{3} E_F
\end{equation}



