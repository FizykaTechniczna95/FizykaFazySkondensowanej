\subsubsection*{Rozwiązanie:}
Jedna dozwolona wartość wektora K przypada na element płaszczyzny o powierzchni:
\\
\\
$P=(\frac{2\pi}{L})^2$
\\
\\
zatem wewnątrz koła (rys 6. str 141 Kittel) o powierzchni:
\\
\\
$P=\pi K^2$
\\
\\
liczba drgań na jednostkowy przedział k wynosi:
\\
\\
\begin{equation}
N=\pi K^2(\frac{L}{2\pi})^2
\end{equation}
\\
\\
\begin{equation}
D(\omega)=\frac{d}{d\omega} \pi K^2(\frac{L}{2\pi})^2=\frac{d}{d\omega}\frac{k^2L^2}{4\pi}=\frac{L^2K}{2\pi}\frac{dK}{d\omega}=\frac{L^2}{2\pi}k\frac{1}{v}=\frac{L^2\omega}{2\pi v^2}
\end{equation}
\\
\\
częstotliwość Debaya:
\\
\\
\begin {equation}
N=\frac{k^2L^2}{4\pi}=\frac{\omega^2L^2}{4\pi v^2}
\end{equation}
\\
\begin{equation}
\omega_D=(\frac{4\pi Nv^2}{L^2})^{\frac{1}{2}}
\end{equation}
