\subsubsection*{Rozwiązanie:}
\textbf{podpunkt a}

fala podłużna: 

\begin{equation}
U_s=U cos(\omega-ska)
\end{equation}

a-odległość pomiędzy atomami,
$\omega$-częstotliwość,
s-pozycja atomu,
$k=\frac{2\pi}{\lambda}$
\\
\\
prędkość:
\\
\begin{equation}
v=\frac{dU_s}{dt}=\frac{d}{dt}(Ucos(\omega t- ska)
\end{equation}
\\
\\
energia kinetyczna:
\\
\begin{equation}
E_k=\frac{1}{2}Mv^2=\frac{1}{2}M(\frac{dU_s}{dt})^2
\end{equation}
\\
\\
całkowita energia kinetyczna fali- suma energi poszczególnych atomów:
\\
\\
\begin {equation}
E_k=\sum\frac{1}{2}M(\frac{dU_s}{dt})^2
\end{equation}
\\
\\
energia potencjalna:
\\
\\
\begin{equation}
E_p=\frac{1}{2}kx^2-> \frac{1}{2}C(U_s-U_{s+1})^2
\end{equation}
\\
c-stała siłowa
\\
\\
całkowita energia potencjalna:
\\
\begin{equation}
E_p=\sum\frac{1}{2}kx^2-> \frac{1}{2}C(U_s-U_{s+1})^2
\end{equation}
\\
\\
całkowita energia:
\\
\begin{equation}
E=\frac{1}{2}\sum M(\frac{dU_s}{dt})^2+\frac{1}{2}C\sum(U_s-U_{s+1})^2
\end{equation}
\\
\\
\\
\textbf{podpunkt b}
\\
\\
$U_s=Ucos(\omega t-ska)$
\\
$U_{s+1}=Ucos(\omega t-(s+1)ka)$
\\
\\
korzystamy z zależności $cos(\alpha - \beta)=cos\alpha cos\beta+sin\alpha sin\beta$
\\
\\
$U_{s+1}=U[cos(\omega t- ska-ka]=U[cos(\omega t-ska)cos(ka)+sin(\omega t-ska)sin(ka)]$
\\
\\
$U_s-U_{s+1}=Ucos(\omega t-ska)-U[cos(\omega t- ska-ka]=U[cos(\omega t-ska)cos(ka)+sin(\omega t-ska)sin(ka)]$
\\
\\
$=Ucos(\omega t-ska)[1-cos(ka)]+sin(\omega t-ska)sin(ka)]$
\\
\\
$E=\frac{1}{2}M(\frac{d}{dt}Ucos(\omega t-ska)^2+\frac{1}{2}C[Ucos(\omega t-ska)[1-cos(ka)]+sin(\omega t-ska)sin(ka)]^2$
\\
\\
$=\frac{1}{2}M[-U\omega sin(\omega t-ska]^2+\frac{1}{2}C[Ucos(\omega t-ska)[1-cos(ka)]+sin(\omega t-ska)sin(ka)]^2$
\\
\\
podstawiamy:
\\
\\
$\omega t-ska -> A$
\\
$ka ->B$
\\
\\
$E=\frac{1}{2}M[-u\omega sinA]^2+\frac{1}{2}CU^2[cosA(1-cosB)+sinAsinB]^2$
\\
\\
$=\frac{1}{2}M\omega^2U^2sin^2A+\frac{1}{2}CU^2[cos^2A(1-cosB)^2+sin^2Asin^2B+2cosA(1-cosB)sinAsinB$
\\
\\
$=\frac{1}{2}M\omega^2U^2sin^2A+\frac{1}{2}CU^2[cos^2A(1-2cosB+cos^2B)+sin^2AB+2cosA(1-cosB)sinAsinB$
\\
\\
$=\frac{1}{2}M\omega^2U^2sin^2A+\frac{1}{2}CU^2[cos^2A-2cos^2AcosB+cos^2Acos^B+sin^2Asin^2B+2cosA(1-cosB)sinAsinB$
\\
\\
$<cos^2>=\frac{1}{2}$
\\
$<sin^2>=\frac{1}{2}$
\\
$<sin cos>=0$
\\
\\
$E=\frac{1}{2}M\omega^2U^2\frac{1}{2}+\frac{1}{2}CU^2[\frac{1}{2}-cosB+\frac{1}{4}+\frac{1}{4}+2*0*sinB-2*0*0]$
\\
\\
$=\frac{1}{4}M\omega^2U^2+\frac{1}{2}CU^2[\frac{1}{2}-cosB+\frac{1}{2}]=\frac{1}{4}M\omega^2U^2+\frac{1}{2}[1-cosB]$
\\
\\
B->ka
\\
\\
$E=\frac{1}{4}M\omega^2U^2+\frac{1}{2}CU^2(1-cos(ka)$
\\
\\
relacja dyspersyjna:
\\
\\
$\omega^2=\frac{2C}{M}(1-cos(ka))$
\\
\\
$\frac{M\omega^2}{2C}=(1-cos(ka))$
\\
\\
$E=\frac{1}{4}M\omega^2U^2+\frac{1}{4}U^2\frac{M\omega^2}{2}$
\\
\\
$E=\frac{1}{2}M\omega^2U^2$