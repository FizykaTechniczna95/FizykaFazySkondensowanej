\section{Zadania}
\subsection{Pierwszy}
\subsubsection{}
\begin{enumerate}
\item Sieć  krystaliczna, węzły sieci, proste sieciowe, płaszczyzny sieciowe, wskaźniki Millera (hkl), Komórka elementarna i  typy układów krystalograficznych
\item Operacje symetrii, grupy punktowe.
\item Sieć prosta a sieć odwrotna. Objętości komórki elementarnej w sieci odwrotnej. Odległości międzypłaszczyznowe. Strefy Brillouina.
\end{enumerate}

\subsubsection{}
Obliczyć objętość komórki elementarnej dla układu regularnego, romboedrycznego, heksagonalnego, jednoskośnego.
\subsubsection{}
Wykaż, że:
\begin{enumerate}
\item dla prostej sieci regularnej o stałej sieciowej $a$, odległość międzypłaszczyznowa \[d^2_{hkl} =\frac{a^2}{h^2+k^2+l^2}\] 
\item obliczyć $\frac{1}{d^2_{hkl}}$ dla układu heksagonalnego oraz rombowego
\end{enumerate}

\subsubsection{}
Struktura diamentu zawiera dwa identyczne atomy w położeniach $000$ i $\frac{1}{4}\frac{1}{4}\frac{1}{4}$ związane z każdym węzłem sieci powierzchniowo centrowanej \textit{(fcc)}. Obliczyć czynnik strukturalny dla tej struktury. Pokaż, że dozwolone odbicia spełniają warunek $h + k + l = 4n$, gdzie wszystkie wskaźniki są parzyste, a $n$ jest dowolna liczbą całkowitą, albo wszystkie składniki są nieparzyste.  

\newpage

\subsection{Drugi}
\subsubsection{}
Energia oddziaływania między dwoma atomami w cząsteczce opisywana jest wzorem:
\[U(r) = -\frac{\alpha}{r^n}+\frac{\beta}{r^m}\]
Pokazać, że $m>n$.
\subsubsection{}
Rozważ liniowy układ $2N$ jonów o ładunku równym na przemian $\pm q$. Załóż, że energia potencjalna odpychania między najbliższymi sąsiadami ma postać $\frac{A}{R^n}$. 
\begin{enumerate}
\item Pokaż, że dla odległości między jonami odpowiadającej stanowi równowagi 
\[U(R_0) = -\frac{2Nq^2\ln(2)}{R_0}\left(1-\frac{1}{n}\right)\]
\item Załóżmy, że kryształ został ściśnięty tak, że $R_0\rightarrow R_0(1-\delta)$. Pokaż, że w wyrażeniu na pracę związaną ze ściśnięciem kryształu największy wkład opisuje człon $\frac{C\delta^2}{2}$ gdzie:
\[C=\frac{(n-1)q^2\ln(2)}{R_0}\]
\end{enumerate}
\subsubsection{}
Obliczyć stałą Madelunga dla kryształu \textit{NaCl}:
\begin{enumerate}
\item przypadek jednowymiarowy (nić krystaliczna \textit{NaCl})
\begin{figure}[h!]
\centering
\includegraphics[scale=0.3]{images/zes2-1}
\end{figure}
\item przypadek dwuwymiarowy (siatka płaska \textit{NaCl})
\begin{figure}[h!]
\centering
\includegraphics[scale=0.2]{images/zes2-2}
\end{figure}
\end{enumerate}
\subsubsection{}
Obliczyć jakie ciśnienie należy przyłożyć do kryształu jonowego, aby odległość między jonami zmniejszyła się o $1$ procent.