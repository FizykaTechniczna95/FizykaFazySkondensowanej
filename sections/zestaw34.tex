\subsection{Trzeci}
\subsubsection{}
Poniższy rysunek przedstawia temperaturową zależność oporu elektrycznego. Określ, czy jest to zależność dla metali czy izolatorów. Opisz proces fizyczny, który opisuje tą zależność w zakresie temperatur: 
a) blisko 0 K, 
b) około 25 K, 
c) około 300 K. 
Oszacuj średnią drogę swobodną i czas w $T = 0$K i $T = 300$K. Przydatne stałe: $n = 10^{23} cm^{-3}$, $m = 10^{27}$kg, $v_f = 108 \frac{cm}{s}$, $e = 4.8 \cdot 10^{10} $esu $(e = 1,6^{19} C)$, $1(\Omega cm)^2 = 9 \cdot 10^{11}$esu.
\input{solutions/31}

\subsubsection{}
\begin{figure}[h]
\centering
\includegraphics[scale=0.7]{images/zes3-2}
\end{figure}
\input{solutions/32}

\subsubsection{}
Rozpatrzyć falę podłużną $u_s=u(0)\cos{(\omega t -sKa)}$, która rozchodzi się w jednoatomowej sieci liniowej składającej się  z atomów o masach $M$ odległych od siebie o $a$; stała siłowa oddziaływania między najbliższymi sąsiadami wynosi $C$.
\begin{itemize}
\item Wykazać, że całkowita energia fali wynosi:
\[E = \frac{1}{2}M \sum_s \left(\frac{du_s}{dt}\right)^2 + \frac{1}{2}C\sum_s(u_s - u_{s+1})^2\]
\item Podstawiając wyrażenie na $u_s$ do powyższego wzoru wykaż, że uśredniona w czasie energia całkowita przypadająca na jeden atom wynosi:
\[\frac{1}{4}M\omega^2 u^2(0) + \frac{1}{2}C(1-cos(Ka))u^2(0) = \frac{1}{2} M \omega^2 u^2(0)\]
\end{itemize}
\subsubsection*{Rozwiązanie:}
\textbf{podpunkt a }


\subsubsection{}
Wyznaczyć podłużny fonon akustyczny oraz widmo optyczne dla sieci liniowej o stałej a zawierającej w komórce dwa jednakowe atomy o masach $M$, których odległość w położeniu równowagi wynosi $\delta < \frac{1}{a}$.
\subsubsection*{Rozwiązanie:}
\subsubsection*{Rozwiązanie:}
\hrulefill


\subsubsection{}
Dana jest sieć:
\begin{figure}[h!]
\centering
\includegraphics[scale=0.9]{images/zes3-5}
\end{figure}
\begin{itemize}
\item Wykazać. że
\[M\frac{d^2u_{lm}}{dt^2} = C\big( (u_{l+1,m} - u{l-1,m} - 2u_{lm}) + (u_{l,m+1} - u_{l,m-1} -2u_{lm}) \big)\]
\item Przyjąć: $u_{lm} = u(0) \exp\big( i(lK_xa + mK_ya - \omega t) \big)$ i wykazać, że:
\[ \omega^2 M= 2C(2-\cos (K_xa) - \cos(K_ya) ) \]
\item Wykazać, że przedział wartości wektora $K$, dla których istnieją niezależne rozwiązania można przyjąć kwadrat o baku $\frac{2\pi}{a}$
\item Dla $Ka <<1$ wykazać, że:
\[ \omega = \left( \frac{Ca^2}{M} \right)^{\frac{1}{2}} \big( K_x^2 + K_y^2 \big) = \left( \frac{Ca^2}{M} \right) K \]
\end{itemize}
\input{solutions/35}

\subsection{Czwarty}
\subsubsection{}
\label{subsubsec:41}
Wyprowadzić wzory na funkcję gęstości stanów dla łańcucha jednoatomowego zakładając, że $\omega = \upsilon \cdot k$. Określić częstotliwość Debye'a.
\subsubsection*{Rozwiązanie:}
\textbf{funkcja gęstości stanów}
D$(\omega)$ jest to liczba modów różnych dragań przypadająca na jednostkowy zakres częstotliwości.
\\
\\
przemieszczenia atomu w drganiach podłuznych i poprzecznych określa zależność:
\\
\\
$U_s ~sin(sKa)$
\\
\\
dobieramy tak wartości K żeby atomy na końcu i początku łańcucha były unieruchomione.
\\
wartości wektora falowego, które są dozwolone:
\\
\\
K= +/-$\frac{2\pi}{L},\frac{4\pi}{L},.....$
\\
Gęstość stanów opisuje wzór:
\\
\\
\begin{equation}
D(\omega)=\frac{dN}{d\omega}
\end{equation}
\\
gdzie N to całkowita liczba modów drgań o wartości wektora falowego mniejszego od k.
\\
\\
$K=\frac{N\pi}{L}$
\\
\\
$N=\frac{KL}{N}$
\\
\\
$D(\omega)=\frac{d}{d\omega}\frac{KL}{\pi}=\frac{L}{\pi}\frac{dK}{d\omega}=\frac{L}{\pi v}=\frac{L K}{\pi \omega}$
\\
\\
gdzie $\frac{dK}{d\omega}$ - prędkość grupowa, $k=\frac{\omega}{v}$
\\
\\
częstotliwość Debaya jest to teoretyczna najwyższa możliwa częstotliwość drgań atmów  wsieci krystalicznej.
\\
\\
$N=\frac{KL}{\pi}=\frac{\omega L}{v\pi}$
\\
\\
$\omega_D=\frac{Nv\pi}{L}$


\hrulefill


\subsubsection{}
\label{subsubsec:42}
Wyprowadzić wzory na funkcję gęstości stanów dla sieci kwadratowej zakładając, że $\omega = \upsilon \cdot k$. Określić częstotliwość Debye'a.
\subsubsection*{Rozwiązanie:}
Jedna dozwolona wartość wektora K przypada na element płaszczyzny o powierzchni:
\\
\\
$P=(\frac{2\pi}{L})^2$
\\
\\
zatem wewnątrz koła (rys 6. str 141 Kittel) o powierzchni:
\\
\\
$P=\pi K^2$
\\
\\
liczba drgań na jednostkowy przedział k wynosi:
\\
\\
\begin{equation}
N=\pi K^2(\frac{L}{2\pi})^2
\end{equation}
\\
\\
\begin{equation}
D(\omega)=\frac{d}{d\omega} \pi K^2(\frac{L}{2\pi})^2=\frac{d}{d\omega}\frac{k^2L^2}{4\pi}=\frac{L^2K}{2\pi}\frac{dK}{d\omega}=\frac{L^2}{2\pi}k\frac{1}{v}=\frac{L^2\omega}{2\pi v^2}
\end{equation}
\\
\\
częstotliwość Debaya:
\\
\\
\begin {equation}
N=\frac{k^2L^2}{4\pi}=\frac{\omega^2L^2}{4\pi v^2}
\end{equation}
\\
\begin{equation}
\omega_D=(\frac{4\pi Nv^2}{L^2})^{\frac{1}{2}}
\end{equation}


\subsubsection{}
Korzystając z wyników zadań \ref{subsubsec:41} i \ref{subsubsec:42} wyprowadzić wzory na molowe ciepło właściwe.
\input{solutions/43}

\subsubsection{}
\label{subsubsec:44}
Znaleźć zależność poziomu Fermiego w temperaturze zera bezwzględnego od gęstości elektronowej $n$:
\[E_F(T=0) = \frac{\hbar^2}{2m}(3n\pi)^{\frac{2}{3}}\]
oraz zależność średniej energii na elektron od energii Fermiego.
\[ \overline{E}(T=0) = \frac{3}{5}E_F \]
\subsubsection*{Rozwiązanie:}
Energia Fermiego dana jest wzorem:
\begin{equation}
\label{eq:E_F}
E_F = \frac{\hbar^2}{2m}k_F^2
\end{equation}
Na każdy element objętości $v_1 = \left(\frac{2\pi}{L}\right)^2$ przypada jeden wektor falowy $k$.\\
Stąd dla objętości $v_2 = \frac{4}{3}\pi k_F^3$ całkowita liczba stanów wynosi($2$ dozwolone stany spinowej liczby kwantowej):
\begin{equation}
N = 2 \frac{v_2}{v_1} = 2 \frac{4\pi k_F^3 L^3}{3\cdot 8 \pi^3} =  \frac{V}{3\pi^2}k_F^3
\end{equation}
czyli
\begin{equation}
\label{eq:k_F}
k_F = \left( \frac{3 \pi^2 N}{V} \right)^{\frac{1}{3}} =  (3\pi n)^{\frac{1}{3}}
\end{equation} 
gdzie gęstość stanów $n = \frac{N}{V}$.\\
Podstawiając wyrażenie (\ref{eq:k_F}) do równania (\ref{eq:E_F}) otrzymuje się:
\begin{equation}
E_F = \frac{\hbar^2}{2m} (2n\pi)^{\frac{2}{3}}
\end{equation} 
\hrulefill
\newline
W przypadku 1D energia elektronu w stanie $n$ wynosi:
\begin{equation}
E_n = \frac{\hbar^2 \pi^2}{2mL^2}n^2
\end{equation}
Dla  $N$ poziomów śrenią energię można wyrazić przez:
\begin{equation}
\overline{E} = 2 \frac{\sum^N E_n}{2N}
\end{equation}
gdzie energia ostatniego elektronu jest enegią Fermiego:
\begin{equation}
\label{eq:E_F2}
E_F = \frac{\hbar^2 \pi^2}{2mL^2} N^2
\end{equation}
Stosując przybliżenie:
\begin{equation}
\sum^N_{n=1} n^2 = \frac{1}{6}(2N^2 + 3N + 1) = \frac{N^3}{3} + \frac{N^2}{2} + \frac{N}{6} \sim \frac{N^3}{3}
\end{equation} 
oraz (\ref{eq:E_F2}), średnia energia elektronu wyraża się przez:
\begin{equation}
\overline{E} = \frac{2}{2N} \frac{\hbar^2 \pi^2}{2mL^2} \sum^N n^2 = \frac{2}{2N} \frac{\hbar^2 \pi^2}{2mL^2} \frac{N^3}{3} = \frac{1}{3} E_F
\end{equation}





\subsubsection{}
Wyprowadzić wzór na funkcję gęstości stanów elektronów swobodnych w przypadku jednowymiarowym.
\subsubsection*{Rozwiązanie:}
Energia fermiego- energia najwyżeszego obsadzonego układu N elektronów:
\\
\\
\begin{equation}
E_f=\frac{\hbar^2}{2m}(\frac{n_f\pi}{L})^2
\end{equation}
\\
\\
$n_f$-najwyższy obsadzony poziom energii
\\
\\
przyjmując, że N jest liczbą parzystą: $2n_f=N$ otzrymujemy:
\\
\\
\begin{equation}
E_f=\frac{\hbar^2}{2m}(\frac{N\pi}{2L})^2
\end{equation}
\\
\begin{equation}
N=\frac{2L}{\hbar \pi}\sqrt{2mE_f}
\end{equation}
\\
\\
$D{\omega}$-gęstość stanów-liczba orbitali przypadająca na jednostkę energii.
\\
\\
\begin{equation}
\frac{dN}{d\omega}=L\frac{\sqrt{2m}}{\pi \hbar}\frac{1}{\sqrt{E_f}}
\end{equation}\hrulefill


\subsubsection{}
Wyprowadzić wzór na funkcję gęstości stanów $g(E)$  gazu elektronowego dla sieci kwadratowej.
\subsubsection*{Rozwiązanie:}
Pole koła: $\pi k_f^2$
\\
\\
element powierzchni: $(\frac{2\pi}{L})^2$
\\
\\
\begin{equation}
N=\frac{L^2k_f^2}{4\pi}
\end{equation}
\\
\begin{equation}
f_f=\frac{4\pi N}{L^2} 
\end{equation}
\\
\begin{equation}
E_f=\frac{\hbar^2}{2m}\frac{4\pi N}{L^2}
\end{equation}
\\
\begin{equation}
N=\frac{mL^2E_f}{2\pi \hbar^2}
\end{equation}
\\
\begin{equation}
\frac{dN}{d\omega}=frac{mL^2}{2\pi \hbar^2}
\end{equation}
\hrulefill


\subsubsection{}
Korzystając z wyników zadania \ref{subsubsec:44} wyprowadzić wzór na molowe ciepło właściwe gazu Fermiego w przypadku jednowymiarowym.
\input{solutions/47}